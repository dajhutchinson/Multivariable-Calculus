\documentclass[11pt,a4paper]{article}

\usepackage[margin=1in, paperwidth=8.3in, paperheight=11.7in]{geometry}
\usepackage{amsfonts}
\usepackage{amsmath}
\usepackage{enumerate}
\usepackage{enumitem}
\usepackage{fancyhdr}
\usepackage{listings}
\usepackage{stmaryrd}
\usepackage[table]{xcolor}

\begin{document}

\pagestyle{fancy}
\setlength\parindent{0pt}
\allowdisplaybreaks

% Counters
\newcounter{question}
\newcounter{qpart}[question]

% commands
\newcommand{\nats}{\mathbb{N}}
\newcommand{\reals}{\mathbb{R}}
\newcommand{\newquestion} {\stepcounter{question}}
\newcommand{\newqpart} {\stepcounter{qpart}}
\newcommand{\question}[1] {\newquestion \ifquestions \textbf{Question \arabic{question}}\\ #1\\ \fi}
\newcommand{\qpart}[1] {\newqpart \ifquestions \textbf{Question \arabic{question}.\arabic{qpart}}\\ #1\\ \fi}
\newcommand{\solution}[1] {\ifsolutions\textbf{My Solution \arabic{question}}\\ #1\\ \fi}
\newcommand{\spart}[1] {\ifsolutions\textbf{My Solution \arabic{question}.\arabic{qpart}}\\ #1\\ \fi}
\newcommand{\doubleplus} {+\kern-1.3ex+\kern0.8ex}
\renewcommand{\headrulewidth}{0pt}

% enviroments
\lstnewenvironment{code}
  {\lstset{mathescape=true,xleftmargin=.1\textwidth}}
  {}

% if
\newif\ifquestions
\questionstrue
%\questionsfalse
\newif\ifsolutions
\solutionstrue
%\solutionsfalse

% Cover page title
\title{Multivariable Calculus - Problem Class 6}
\author{Dom Hutchinson}
\date{\today}
\maketitle

% Header
\fancyhead[L]{Dom Hutchinson}
\fancyhead[C]{Multivariable Calculus - Problem Class 6}
\fancyhead[R]{\today}

\setcounter{question}{1}
\question{
Calculate
$$\int_C\textbf{r}\circ d\textbf{r}$$
where $C$ is any curve connecting the point $\textbf{r}_1$ to $\textbf{r}_2$.
}

\solution{
\underline{Method 1}
\[\begin{array}{rcl}
\int_C\textbf{r}\circ d\textbf{r}&=&\int_Cd\left(\frac{1}{2}\textbf{r}\circ\textbf{r}\right)\\
&=&\left[\frac{1}{2}\textbf{r}\circ\textbf{r}\right]_{\textbf{r}_2}^{\textbf{r}_1}\\
&=&\frac{1}{2}|\textbf{r}_2|^2-\frac{1}{2}|\textbf{r}_1|^2
\end{array}\]
\underline{Method 2}
\[\begin{array}{rrcl}
\mathrm{Let}&f&=&\frac{1}{2}\textbf{r}\circ\textbf{r}\\
\implies&\nabla f&=&\textbf{r}\\
\mathrm{Since}&\nabla\left(\frac{1}{2}\textbf{r}\circ\textbf{r}\right)&=&\frac{1}{2}\nabla(r^2)\\
&&=&\frac{2r}{2}\nabla r\\
&&=&r\left(\frac{\textbf{r}}{r}\right)\\
&&=&\textbf{r}
\end{array}\]
}

\setcounter{question}{3}
\question{
Let $\phi$ be a scalar field. use the divergence theorem to show that
$$\int_V\nabla\phi dV=\int_{\partial V}\phi\hat{\textbf{n}}dS$$
}

\solution{
The Divergence Theorem states
$$\int_V\nabla\circ\textbf{f}dV=\int_{\partial V}\textbf{f}\circ d\textbf{S}$$
Let $\textbf{f}=\phi(\textbf{r})\textbf{a}$ where $\textbf{a}\in\reals^3$ is constant.\\
Consider
\[\begin{array}{rcl}
\nabla\circ\textbf{f}&=&\nabla\circ(\phi(\textbf{r})\textbf{a})\\
&=&(\nabla\phi)\circ\textbf{a}+\phi(\nabla\circ\textbf{a})\\
&=&(\nabla\phi)\circ\textbf{a}
\end{array}\]
By applying the divergence theorem
$$\textbf{a}\circ\int_V\nabla\phi dV = \textbf{a}\circ\int_{\partial V}\phi d\textbf{S}$$
This holds $\forall\ \textbf{a}$ so
$$\int_V\nabla\phi dV=\int_{\partial V}\phi d\textbf{S}=\int_{\partial V}\phi\hat{\textbf{n}}dS$$
}

\newquestion
\qpart{
A vector field $\textbf{F}$ is given by $\textbf{F}(x,y,z)=(xy,yz,xz)$. Calculate $\nabla\circ\textbf{F}$.
}

\spart{
$$\nabla\circ\textbf{F}(x)=x+y+z$$
}

\qpart{
The volume, $V$ of a tetrahedron is bounded by four triangular surfaces formed by the intersection of the planes $x=0$, $y=0$, $z=0$ and $x+y+z=1$.\\
Calculate the volume integral
$$\int_V\nabla\circ\textbf{F}dV$$
}

\spart{
\[\begin{array}{rcl}
\int_V\nabla\circ\textbf{F}dV&=&\displaystyle{\int_{x=0}^{x=1}\int_{y=0}^{y=1-x}\int_{z=0}^{z=1-x-y}(x+y+z)dzdydx}\\
&=&\displaystyle{\int_{x=0}^{x=1}\int_{y=0}^{y=1-x}\left((x+y)(1-x-y)+\frac{1}{2}(1-x-y)^2\right)dydz}\\
&=&...%TODO
\end{array}\]
}

\qpart{
State the divergence theorem and use it ot calculate the value of the integral in 5.2 by an independent method.
}

\spart{
$\int_V\nabla\circ\textbf{F}dV=\int_{\partial V}\textbf{F}\circ d\textbf{S}$\\
where $\partial V=S_1\cup S_2\cup S_3\cup S_4$.\\
Let $S_1$ be the volume surface in $x=0$ plane\\
$\implies\textbf{F}(x,y,z)=(0,yz,0)$ and $\partial\textbf{S}=\hat{\textbf{n}}dS=-\hat{\textbf{x}}dS$\\
$\implies\textbf{F}\circ d\textbf{S}=0$ on $S_1$.\\
Similarly $\textbf{F}\circ d\textbf{S}=0$ on $S_2$ \& $S_3$.\\
$\implies\int_{\partial V}\textbf{F}\circ d\textbf{S}=\int_{S_4}\textbf{F}\circ d\textbf{S}$.\\
Let $D=\{(x,y)|0<x<1,0<y<1-x\}$ and $\textbf{s}(x,y)=(x,y,1-x-y)$ for $(x,y)\in D$.\\
Now mechanical %TODO
}


\question{
If $\textbf{f}(\textbf{r})=(0,x,0)$ and $\textbf{g}(\textbf{r})=(-y,0,0)$ show that, for any closed curve $C\in\reals^3$
$$\int_C\textbf{f}\circ d\textbf{r}=\int_C\textbf{g}\circ d\textbf{r}$$
}

\solution{
\underline{Method 1}\\
Consider $\int_C(\textbf{f}-\textbf{g})\circ d\textbf{r}$\\
We want to show $\textbf{f}-\textbf{g}=\nabla f$ so that we can use the Fundamental Theorem of Calculus.\\
This requires $\frac{\partial f}{\partial x}=g\implies f=yx$ and $\frac{\partial f}{\partial y}=x\implies f=xy$.\\
These both produce the same result, so hold.\\
Set $\nabla(x,y)=\textbf{f}-\textbf{g}$.
$$\implies\int_C(\textbf{f}-\textbf{g})d\textbf{r}=\int_C\nabla \textbf{f}\circ d\textbf{r}=0$$
Since it is a closed loop.\\
\underline{Method 2}\\
Set $\textbf{A}=\textbf{f}-\textbf{g}=(y,x,0)$
\[\begin{array}{rrcll}
\implies&\int_C\textbf{A}\circ d\textbf{r}&=&\int_C\nabla\times\textbf{A}\circ d\textbf{S}&\mathrm{Stokes'\ Theorem}\\
&\nabla\times\textbf{A}&=&\begin{vmatrix}
\hat{\textbf{x}} & \hat{\textbf{y}} & \hat{\textbf{z}}\\
\partial x & \partial y & \partial z\\
y & x & 0
\end{vmatrix}\\
&&=&(0,0,1-1)\\
&&=&\textbf{0}\\
\implies&\int_C\textbf{A}\circ d\textbf{r}&=&\int_S\nabla\times\textbf{A}\circ d\textbf{S}\\
&&=&\int_S 0 d\textbf{S}\\
&&=&0
\end{array}\]
}

\end{document}
