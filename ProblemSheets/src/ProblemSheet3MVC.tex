\documentclass[11pt,a4paper]{article}

\usepackage[margin=1in, paperwidth=8.3in, paperheight=11.7in]{geometry}
\usepackage{amsfonts}
\usepackage{amsmath}
\usepackage{enumerate}
\usepackage{enumitem}
\usepackage{fancyhdr}
\usepackage{listings}
\usepackage{stmaryrd}
\usepackage[table]{xcolor}

\begin{document}

\pagestyle{fancy}
\setlength\parindent{0pt}
\allowdisplaybreaks

% Counters
\newcounter{question}
\newcounter{qpart}[question]

% commands
\newcommand{\nats}{\mathbb{N}}
\newcommand{\real}{\mathbb{R}}
\newcommand{\newquestion} {\stepcounter{question}}
\newcommand{\newqpart} {\stepcounter{qpart}}
\newcommand{\question}[1] {\newquestion \ifquestions \textbf{Question \arabic{question}}\\ #1\\ \fi}
\newcommand{\qpart}[1] {\newqpart \ifquestions \textbf{Question \arabic{question}.\arabic{qpart}}\\ #1\\ \fi}
\newcommand{\solution}[1] {\ifsolutions\textbf{My Solution \arabic{question}}\\ #1\\ \fi}
\newcommand{\spart}[1] {\ifsolutions\textbf{My Solution \arabic{question}.\arabic{qpart}}\\ #1\\ \fi}
\newcommand{\Spart}[1] {\ifsolutions\textbf{My Solution \arabic{question}.\arabic{qpart}} #1 \fi}
\newcommand{\doubleplus} {+\kern-1.3ex+\kern0.8ex}
\renewcommand{\headrulewidth}{0pt}

% enviroments
\lstnewenvironment{code}
  {\lstset{mathescape=true,xleftmargin=.1\textwidth}}
  {}

% if
\newif\ifquestions
\questionstrue
%\questionsfalse
\newif\ifsolutions
\solutionstrue
%\solutionsfalse

% Cover page title
\title{Multivariable Calculus - Problem Sheet 3}
\author{Dom Hutchinson}
\date{\today}
\maketitle

% Header
\fancyhead[L]{Dom Hutchinson}
\fancyhead[C]{Multivariable Calculus - Problem Sheet 3}
\fancyhead[R]{\today}

\setcounter{question}{3}
\qpart{
Let $\phi(\textbf{r})=f(r)$, where $f$ is a function of a single variable, $r=|\textbf{r}|$. Computer $\Delta\phi$.
}

\Spart{
\[\begin{array}{rcl}=
\Delta\phi&=&\sum_{\alpha=1}^3\frac{\partial^2}{\partial x_\alpha^2}f(r)\\
&=&\sum_{\alpha=1}^3\frac{\partial}{\partial x_\alpha}\left(\frac{\partial}{\partial x_\alpha}f(r)\right)\\
&=&\sum_{\alpha=1}^3\frac{\partial}{\partial x_\alpha}\left(\frac{\partial r}{\partial x_\alpha}\frac{\partial f}{\partial r}\right)\\
&=&f'\sum_{\alpha=1}^3\frac{\partial^2}{\partial x_\alpha^2}r\\
&=&f'\sum_{\alpha=1}^3\frac{\partial}{\partial x_\alpha}\frac{x_\alpha}{r}\\
&=&f'\sum_{\alpha=1}^3\left[\frac{1}{r}-x_\alpha\left(\frac{x_\alpha}{r^3}\right)\right]\\
&=&\frac{f'}{r}\sum_{\alpha=1}^3\left[1-\frac{x_\alpha^2}{r^2}\right]\\
&=&\frac{f'}{r}\left[3-\frac{r^2}{r^2}\right]\\
&=&2\frac{f'}{r}
\end{array}\]
}

\qpart{
Let $\pmb{\mu}\in\real^3$ be a nonzero vector. Compute $\Delta\dfrac{\pmb{\mu}\circ\textbf{r}}{r^3}$.
}

\Spart{
\[\begin{array}{rcl}
\Delta\dfrac{\pmb{\mu}\circ\textbf{r}}{r^3}&=&\nabla\circ\left(\nabla\dfrac{\pmb{\mu}\circ\textbf{r}}{r^3}\right)\\
\nabla\dfrac{\pmb{\mu}\circ\textbf{r}}{r^3}&=&\left(
\dfrac{\mu_1x}{r^3},\ \dfrac{\mu_2y}{r^3},\ \dfrac{\mu_3z}{r^3}
\right)\\
&=&\left(
\dfrac{\mu_1r^3-(\mu_1x)(3xr)}{r^6},\ \dfrac{\mu_2r^3-(\mu_2y)(3yr)}{r^6},\ \dfrac{\mu_3r^3-(\mu_3z)(3zr)}{r^6}
\right)\\
\nabla\circ\left(\nabla\dfrac{\pmb{\mu}\circ\textbf{r}}{r^3}\right)&=&\dfrac{\mu_1(-4x)r^5-\mu_1(r^2-3x^2)5xr^3}{r^{10}}
+\dfrac{\mu_2(-4y)r^5-\mu_2(r^2-3y^2)5yr^3}{r^{10}}\\
&+&\dfrac{\mu_3(-4z)r^5-\mu_3(r^2-3z^2)5zr^3}{r^{10}}\\
&=&\frac{3}{r^7}\left(
\mu_1x(5x^2-3r^2)+\mu_2y(5y^2-3r^2)+\mu_3z(5z^2-3r^2)
\right)\\
&=&\frac{3}{r^7}\mu_ix_i(5x_i^2-3r^2)
\end{array}\]
}

\question{
This question concerns the transformation to elliptical coordinates $(\mu,\nu)$ given by the relation
$$x=a\cosh\mu\cos\nu\quad y=a\sinh\mu\sin\nu$$
where $mu\in[0,\infty),\ \nu\in[0,2\pi)$.
}

\qpart{
Show that curves of constant $\mu$ correspond to ellipses in the $(x,y)$ plane and that curves of constant $\nu$ are hyperbolae.
}

\spart{
Set $\mu$ to be constant.\\
Then $\exists$ constants $b,c\in\real$ such that $\cosh\mu=b$ \& $\sinh\mu=c$.\\
Then $x=ab\cos\nu$ \& $y=ac\sin\nu$.\\
The general formula for an ellipse is
$$\dfrac{x^2}{\alpha^2}+\dfrac{y^2}{\beta^2}=1$$
Set $\alpha=ab$ \& $\beta=ac$.
\[\begin{array}{rcl}
\implies\frac{x^2}{\alpha^2}+\frac{y^2}{\beta^2}&=&\frac{(ab\cos\nu)^2}{(ab)^2}+\frac{(ac\sin\nu)^2}{(ac)^2}\\
&=&\frac{a^2b^2\cos^2\nu}{a^2b^2}+\frac{a^2c^2\sin^2\nu}{a^2c^2}\\
&=&\cos^2\nu+\sin^2\nu\\
&=&1
\end{array}\]
Thus, for a constant $\mu$, this curves corresponds to the general formula of an ellipse.\\

Set $\nu$ to be constant.\\
Then $\exists$ constants $d, e\in\real$ such that $\cos\nu=d$ \& $\sin\nu=e$.\\
Then $x=ad\cosh\mu$ \& $y=ae\sinh\mu$.\\
The general formula for a hyperbola is
$$\frac{y^2}{\gamma^2}-\frac{x^2}{\delta^2}=1$$
Set $\gamma=ae$ \& $\delta=ad$.
\[\begin{array}{rcl}
\frac{y^2}{\gamma^2}-\frac{x^2}{\delta^2}&=&\frac{(ae\sinh\mu)^2}{(ae)^2}-\frac{(ad\cosh\mu)^2}{(ad)^2}\\
&=&\sinh^2\mu-\cosh^2\mu\\
&=&1
\end{array}\]
Thus, for a constant $\nu$, this curve corresponds to the general formula for a hyperbola.
}

\qpart{
Derive a basis $\hat{\pmb{\mu}},\ \hat{\pmb{\nu}}$ for elliptical coordinates and, in doing so, show that the scal factors are
$$h_\mu=h_\nu=a\sqrt{\sinh^2\mu+\sin^2\nu}$$
Also confirm that $\hat{\pmb{\mu}}$ and $\hat{\pmb{\nu}}$ are orthogonal.
}

\Spart{
\[\begin{array}{rcl}
\textbf{r}&=&(a\cosh\mu\cos\nu,a\sinh\mu\sin\nu)\\

\\h_\mu&=&|\frac{\partial\textbf{r}}{\partial\mu}|\\
&=&\sqrt{(a\sinh\mu\cos\nu)^2+(a\cosh\mu\sin\nu)^2}\\
&=&a\sqrt{\sinh^2\mu\cos^2\nu+\cosh^2\mu\sin^2\nu}\\
&=&a\sqrt{\sinh^2\mu\cos^2\nu+(1+\sinh^2\mu)\sin^2\nu}\\
&=&a\sqrt{\sinh^2\mu(\cos^2\nu+\sin^2\nu)+\sin^2\nu}\\
&=&a\sqrt{\sinh^2\mu+\sin^2\nu}\\

\\h_\nu&=&|\frac{\partial\textbf{r}}{\partial\nu}|\\
&=&\sqrt{(-a\cosh\mu\sin\nu)^2+(a\sinh\mu\cos\nu)^2}\\
&=&a\sqrt{\cosh^2\mu\sin^2\nu+\sinh^2\mu\cos^2\nu}\\
&=&a\sqrt{\sinh^2\mu+\sin^2\nu}\\

\\\hat{\pmb{\mu}}&=&\frac{1}{h_\mu}\frac{\partial\textbf{r}}{\partial\mu}\\
&=&\frac{1}{a\sqrt{\sinh^2\mu+\sin^2\nu}}(a\sinh\mu\cos\nu,\ a\cosh\mu\sin\nu)\\
&=&\frac{1}{\sqrt{\sinh^2\mu+\sin^2\nu}}(\sinh\mu\cos\nu,\ \cosh\mu\sin\nu)\\

\\\hat{\pmb{\nu}}&=&\frac{1}{h_\nu}\frac{\partial\textbf{r}}{\partial\nu}\\
&=&\frac{1}{a\sqrt{\sinh^2\mu+\sin^2\nu}}(-a\cosh\mu\sin\nu,\ a\sinh\mu\cos\nu)\\
&=&\frac{1}{\sqrt{\sinh^2\mu+\sin^2\nu}}(-\cosh\mu\sin\nu,\ \sinh\mu\cos\nu)\\

\\\hat{\pmb{\mu}}\circ\hat{\pmb{\nu}}&=&\frac{1}{\sinh^2\mu+\sin^2\nu}\left[(\sinh\mu\cos\nu)(-\cosh\mu\sin\nu)+(\cosh\mu\sin\nu)(\sinh\mu\cos\nu)\right]\\
&=&\frac{1}{\sinh^2\mu+\sin^2\nu}[0]\\
&=&0
\end{array}\]
Since $\hat{\pmb{\mu}}\circ\hat{\pmb{\nu}}=0$ then $\hat{\pmb{\mu}}$ \& $\hat{\pmb{\nu}}$ are orthogonal.\\
}

\qpart{
Calculate the Jacobian determinant. Is the mapping of coordinates always invertible? If not, when is it non-invertible?
}

\Spart{
\[\begin{array}{rcl}
J_\textbf{r}&=&\begin{vmatrix}
h_\mu[\hat{\pmb{\mu}}]_1 & h_\nu[\hat{\pmb{\nu}}]_1\\
h_\mu[\hat{\pmb{\mu}}]_2 & h_\nu[\hat{\pmb{\nu}}]_2\\
\end{vmatrix}\\
&=&\begin{vmatrix}
a\sinh\mu\cos\nu & -a\cosh\mu\sin\nu\\
a\cosh\mu\sin\nu & a\sinh\mu\cos\nu
\end{vmatrix}\\
&=&(a\sin\mu\cos\nu)^2+(a\cosh\mu\sin\nu)^2\\
&=&a^2\left[\sinh^2\mu\cos^2\nu+\cosh^2\mu\sin^2\nu]\right]\\
&=&a^2\left[\sinh^2\mu+\sin^2\nu\right]
\end{array}\]
This coordinate mapping is not always invertible.\\
It is not invertible if $(\mu,\nu)=(0,\pi n)\ \forall\ n\in\nats$.\\
}

\qpart{
Express $\nabla f$ in elliptical coordinates.
}

\Spart{
\[\begin{array}{rcl}
\nabla f(x,y)&=&\nabla f(a\cosh\mu\cos\nu,\ a\sinh\mu\sin\nu)\\
&=&\frac{\hat{\pmb{\mu}}}{h_\mu}\frac{\partial f}{\partial \mu}+\frac{\hat{\pmb{\nu}}}{h_\nu}\frac{\partial f}{\partial \nu}\\
&=&\frac{1}{a(\sinh^2\mu+\sin^2\nu)}\left[(a\sinh\mu\cos\nu,\ a\cosh\mu\sin\nu)\frac{\partial f}{\partial\mu}+(-a\cosh\mu\sin\nu,\ a\sinh\mu\cos\nu)\frac{\partial f}{\partial\nu}\right]\\
&=&\frac{1}{\sinh^2\mu+\sin^2\nu}\left[(\sinh\mu\cos\nu,\ \cosh\mu\sin\nu)\frac{\partial f}{\partial\mu}+(-\cosh\mu\sin\nu,\ \sinh\mu\cos\nu)\frac{\partial f}{\partial\nu}\right]
\end{array}\]
}



\qpart{
Find $\Delta f$ in elliptical coordiantes.
}

\Spart{
}
\[\begin{array}{rcl}
\Delta f&=&\nabla\circ(\nabla f)\\
&=&\frac{\partial}{\partial\mu}(\nabla f_1)+\frac{\partial}{\partial\nu}(\nabla f_2)\\
&=&\frac{\partial}{\partial\mu}\left[\frac{1}{(\sinh^2\mu+\sin^2\nu)}(\sinh\mu\cos\nu\frac{\partial f}{\partial\mu}-\cosh\mu\sin\nu\frac{\partial f}{\partial \mu})\right]\\
&+&\frac{\partial}{\partial\nu}\left[\frac{1}{(\sinh^2\mu+\sin^2\nu)}(\cosh\mu\sin\nu\frac{\partial f}{\partial \nu}+\sinh\mu\cos\nu\frac{\partial f}{\partial \nu})\right]\\
&=&\bigg[\frac{2\sinh\mu\cosh\mu}{(\sinh^2\mu+\sin^2\nu)^2}\left(\sinh\mu\cos\nu\frac{\partial f}{\partial\mu}-\cosh\mu\sin\nu\frac{\partial f}{\partial \mu}\right)\\
&+&\frac{1}{(\sinh^2\mu+\sin^2\nu)}\left(\cosh\mu\cos\nu\frac{\partial f}{\partial\mu}+\sinh\mu\cos\nu\frac{\partial^2f}{\partial\mu^2}-\sinh\mu\sin\nu\frac{\partial f}{\partial \mu}-\cosh\mu\sin\nu\frac{\partial^2f}{\partial\mu^2}\right)\bigg]\\
&+&\bigg[\frac{-2\sin\nu\cos\nu}{(\sinh^2\mu+\sin^2\nu)^2}(\cosh\mu\sin\nu\frac{\partial f}{\partial\nu}+\sinh\mu\cos\nu\frac{\partial f}{\partial\nu})\\
&+&\frac{1}{(\sinh^2\mu+\sin^2\nu)}(\cosh\mu\cos\nu\frac{\partial f}{\partial\nu}+\cosh\mu\sin\nu\frac{\partial^2f}{\partial\nu^2}-\sinh\mu\sin\nu\frac{\partial f}{\partial \nu}+\sinh\mu\cos\nu\frac{\partial^2f}{\partial\mu^2})\bigg]\\
&=&\left[\frac{(2\sinh^2\mu\cosh\mu\cos\nu-2\sinh\mu\cosh^2\mu\sin\nu)+(\sinh^2\mu+\sin^2\nu)(\cosh\mu\cos\nu-\sinh\mu\sin\nu)}{(\sinh^2\mu+\sin^2\nu)^2}\right]\frac{\partial f}{\partial\mu}\\
&+&\left[\frac{(-2\cosh\mu\sin^2\nu\cos\nu-2\sinh\mu\sin\nu\cos^2\nu)+(\sinh^2\nu+\sin^2\nu)(\cosh\mu\cos\nu-\sinh\mu\sin\nu)}{(\sinh^2\mu+\sin^2\nu)^2}\right]\frac{\partial f}{\partial\nu}\\
&+&\left[\frac{\sinh\mu\cos\nu-\cosh\mu\sin\nu}{(\sinh^2\mu+\sin^2\nu)}\right]\frac{\partial^2f}{\partial\mu^2}+\left[\frac{\cosh\mu\sin\nu+\sinh\mu\cos\nu}{(\sinh^2\mu+\sin^2\nu)}\right]\frac{\partial^2f}{\partial\nu^2}\\
&=&\left[\frac{(2\sinh^2\mu\cosh\mu\cos\nu-2\sinh\mu\cosh^2\mu\sin\nu)+(\sinh^2\mu+\sin^2\nu)(\cosh\mu\cos\nu)-(\cosh^2\mu-\cos^2\nu)(\sinh\mu\sin\nu)}{(\sinh^2\mu+\sin^2\nu)^2}\right]\frac{\partial f}{\partial\mu}\\
&+&\left[\frac{(-2\cosh\mu\sin^2\nu\cos\nu-2\sinh\mu\sin\nu\cos^2\nu)+(\sinh^2\mu+\sin^2\nu)(\cosh\mu\cos\nu)-(\cosh^2\mu-\cos^2\nu)(\sinh\mu\sin\nu)}{(\sinh^2\mu+\sin^2\nu)^2}\right]\frac{\partial f}{\partial\nu}\\
&+&\left[\frac{a(\hat{\pmb{\mu}}_1+\hat{\pmb{\nu}}_1)}{h_\mu}\right]\frac{\partial^2 f}{\partial\mu^2}+\left[\frac{a(\hat{\pmb{\mu}}_2+\hat{\pmb{\nu}}_2)}{h_\mu}\right]\frac{\partial^2 f}{\partial\nu^2}\\
&=&\left[\frac{-\sinh^2\mu\cosh\mu\cos\nu+\sinh\mu\cosh^2\mu\sin\nu+\cosh\mu\sin^2\nu\cos\nu+\sinh\mu\sin\nu\cos^2\nu}{(\sinh^2\mu+\sin^2\nu)^2}\right]\frac{\partial f}{\partial\mu}\\
&+&\left[\frac{-\cosh\mu\sin^2\nu\cos\nu-\sinh\mu\sin\nu\cos^2\nu+\sinh^2\mu\cosh\mu\cos\nu-\sinh\mu\cosh^2\mu\sin\nu}{(\sinh^2\mu+\sin^2\nu)^2}\right]\frac{\partial f}{\partial\nu}\\
&+&\left[\frac{a(\hat{\pmb{\mu}}_1+\hat{\pmb{\nu}}_1)}{h_\mu}\right]\frac{\partial^2 f}{\partial\mu^2}+\left[\frac{a(\hat{\pmb{\mu}}_2+\hat{\pmb{\nu}}_2)}{h_\mu}\right]\frac{\partial^2 f}{\partial\nu^2}\\
&=&\left[\frac{3a^3\sinh\mu\cosh\mu(\hat{\pmb{\mu}}_1-\hat{\pmb{\mu}}_2)+a^3\sin\nu\cos\nu(\hat{\pmb{\mu}}_1+\hat{\pmb{\mu}}_2)}{h_\mu^3}\right]\frac{\partial f}{\partial\mu}
+\left[\frac{a^3\sinh\mu\cosh\mu(\hat{\pmb{\nu}}_1+\hat{\pmb{\nu}}_2)+a^3\sin\nu\cos\nu(\hat{\pmb{\nu}}_1-\hat{\pmb{\nu}}_2)}{h_\nu^3}\right]\frac{\partial f}{\partial\nu}\\
&+&\left[\frac{a(\hat{\pmb{\mu}}_1+\hat{\pmb{\nu}}_1)}{h_\mu}\right]\frac{\partial^2 f}{\partial\mu^2}+\left[\frac{a(\hat{\pmb{\mu}}_2+\hat{\pmb{\nu}}_2)}{h_\mu}\right]\frac{\partial^2 f}{\partial\nu^2}\\
\end{array}\]

\end{document}
